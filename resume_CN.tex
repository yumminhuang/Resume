% !TEX TS-program = xelatex
% !TEX encoding = UTF-8 Unicode
% !Mode:: "TeX:UTF-8"
\documentclass{resume}
\usepackage{zh_CN-fonts} % Simplified Chinese Support using system fonts
\usepackage{linespacing_fix} % disable extra space before next section
\usepackage{cite}
\usepackage[colorlinks,linkcolor=blue,anchorcolor=blue,citecolor=green,urlcolor=blue]{hyperref}


\begin{document}
\pagenumbering{gobble} % suppress displaying page number

\name{ 黄亚铭 }

% {E-mail}{mobilephone}{Github}
% be careful of _ in emaill address
\contactInfo{\href{mailto:yumminhuang@gmail.com}{yumminhuang@gmail.com}}
{(+86) 12345678901}
{\href{https://github.com/yumminhuang}{yumminhuang}}

% Education

\section{教育经历}
\datedsubsection{\textbf{东北大学},波士顿,美国}{2013年9月 --- 2015年12月}
\datedline{\textit{计算机科学硕士学位}}{GPA: 3.875/4.0}
\textbf{部分课程项目}
\begin{itemize}
\item \datedline{MapReduce 项目}{2015年10月 --- 12月}
\begin{itemize}
\item 分析旧金山市犯罪记录,并使用朴素贝叶斯算法预测一个案件的犯罪类型;
\item 在 AWS EMR 上运行 Hadoop 和 Spark 程序。
\end{itemize}

\item \datedline{计算机网络项目}{2014年1月 --- 4月}
\begin{itemize}
\item \textit{Raw Sockets} --- 使用 Python 的 Row Sockets 构建 TCP/IP 栈;
\item \textit{Roll Your Own CDN} --- 基于 AWS EC2 搭建一个简易的「内容分发网络(CDN)」。
\end{itemize}

\item \datedline{数据库项目}{2013年9月 --- 12月}
\begin{itemize}
\item 使用 JPA 为 MySQL 搭建 OData 服务(一款用来访问数据库的 RESTful API)。
\end{itemize}
\end{itemize}

\datedsubsection{\textbf{暨南大学},广州}{2009年9月 --- 2013年7月}
\datedline{\textit{软件工程工学学士学位}}{平均分:84.6/100}
\textbf{毕业论文}
\begin{itemize}
\item 《可视化的Java多线程程序错误定位工具》
\begin{itemize}
\item 一个动态的错误定位工具,用来定位 Java 多线程程序当作的并发错误;
\item 使用 \textit{Soot} 分析 Java 字节码,进行插装。
\end{itemize}
\end{itemize}


%
%Experience
%

%\datedsubsection{\textbf{xxx Projects}}{Jan. 2015 -- Present}
%\role{C, Python, Django, Linux}{Individual Projects, collaborated with xxx}
%Brief introduction: xxx
%\begin{itemize}
%  \item Implemented xxx feature
%  \item Optimized xxx 5\%
%  \item xxx
%\end{itemize}

\section{职业经历}
\datedsubsection{\textbf{BitSight},剑桥,美国}{2015年1月 --- 8月}
\role{实习运维工程师}{}
\begin{itemize}
\item 为 Ubuntu 集群设计并构建了一套安全更新管理系统:使用 Sensu 报告可安装更新,利用 Jenkins  安装更新;
\item 改进公司持续集成框架:从 Jenkins 获取测试结果,将测试结果和测试覆盖率生成总结报告并发送到 BitBucket 页面;
\item 使用 Jenkins 和 Fabric 创建了一个自动化工作流,使用 Django API 来备份和还原测试数据库;
\item 改进了一些开源项目(包括 automated-ebs-snapshots, BitBucket API, sensu-plugin 等);
\item 部署并对比几款日志管理系统(ELK, Logentries, Sumologic):使用 Chef 部署,创建控制面板和常用搜索。
\end{itemize}

\datedsubsection{\textbf{Locately},波士顿,美国}{2014年7月 --- 8月}
\role{实习后端开发工程师}{}
\begin{itemize}
\item 实现 AWS 的 \emph{Auto Scaling} 功能,帮助公司节每年省运营成本 \$3000;
\item 开发 Apache access log 监视器,将 EC2 实例上收集的数据发送到 AWS CloudWatch,用来监控服务器性能;
\item 重写测试代码,使得 API 测试可以并行运行,从而使测试速度提高 17\%。
\end{itemize}

\datedsubsection{\textbf{阿斯顿大学},伯明翰,英国}{2012年3月 --- 4月}
\role{实习软件开发}{}
\begin{itemize}
\item 开发了一款名为 \textit{Scenario Capture} 的 Eclipse 插件,可以在图形界面下操作,并自动生成 JUnit 测试代码;
\item 使用 JUnit 测试代码。
\end{itemize}

% Skills

\section{专业技能}
% increase linespacing [parsep=0.5ex]
\begin{itemize}[parsep=0.5ex]
  \item 语言:Python, Java, Ruby, Shell, Racket
  \item 工具:Chef, Docker, Git, Jenkins, Jira, SQL, UML
  \item 框架:Django, Hadoop
  \item 方法:Agile, Scrum
\end{itemize}

%
%Honors & Awards
%

% \section{\faHeartO\ Honors \& Awards}

%
% Publications
%

%\section {Publication}

% Reference Test
%\datedsubsection{\textbf{Paper Title\cite{zaharia2012resilient}}}{May. 2015}
%An xxx optimized for xxx\cite{verma2015large}
%\begin{itemize}
%  \item main contribution
%\end{itemize}


% \section{Miscellaneous}
% \begin{itemize}[parsep=0.5ex]
% \item \faHome:  \url{http://yumminhuang.github.io}
% \item \faLinkedin: \url{https://www.linkedin.com/pub/yaming-huang/5b/932/6a0}
% \end{itemize}

\end{document}
