\documentclass{resume}

% Use default fonts on Ubuntu
\usepackage[UTF8, heading = false, scheme = plain]{ctex}

% Font sets for MacTex on OS X
% \setCJKmainfont[
% BoldFont=STSongti-SC-Bold,
% ItalicFont=STKaiti,
% SmallCapsFont=STHeiti
% ]{STSong}
% \setCJKsansfont[BoldFont=STHeiti]{STXihei}
% \setCJKmonofont{STFangsong}

\begin{document}

%%%%%%%%%%%%%%%%%%%%%%%%%%%%%%%%%%%%%%%%%%%%%%%%%%%%%%%%%%%%%%%%%%%%%%%%%%%%%%%%
% Name
\contact{黄亚铭}
{\href{mailto:yumminhuang@gmail.com}{yumminhuang@gmail.com}}
{(+86) 123-4567-8901}
{\href{https://github.com/yumminhuang}{yumminhuang}}
{\href{https://www.linkedin.com/in/yaming-huang-6a09325b/zh-cn}{黄亚铭}}
{}

%%%%%%%%%%%%%%%%%%%%%%%%%%%%%%%%%%%%%%%%%%%%%%%%%%%%%%%%%%%%%%%%%%%%%%%%%%%%%%%%
% Experience
\header{职业经历}\begin{body}
	\company{Thoughtworks}
	{上海}
	{2019年9月 --- 至今}
	{高级咨询师}
	\begin{itemize}
		%	\item \project{澳大利亚某公司支付业务系统}{2020年2月 --- 至今} \textbf{DevOps Tech Lead}
		%	\begin{itemize}[noitemsep,topsep=0pt]
		%		\item 基于 CIS Benchmark 安全规范,修复生产环境在内的 3 个 AWS 帐户安全漏洞。
		%		\item PCI-DSS 合规
		%	\end{itemize}
		\item \project{新加坡某政府部门业务系统}{2019年9月 --- 2019年12月} \textbf{DevOps Specialist}
		\begin{itemize}[noitemsep,topsep=0pt]
			\item 在政府私有机房内搭建基于 GoCD 的部署平台,实现自动化部署,同时满足政府严格的安全合规需求。
			\item 完善系统监控和日志收集,实现对系统内基础设施、微服务和第三方集成的全面监控和日志诊断。
		\end{itemize}
	\end{itemize}
	\company{Thoughtworks}
	{北京}
	{2017年8月 --- 2019年9月}
	{咨询师}
	\begin{itemize}
		\item \project{某国际知名汽车品牌中国区 4S 店移动销售平台}{} \textbf{DevOps Tech Lead}
		\begin{itemize}[noitemsep,topsep=0pt]
			\item 在百人开发团队提供 DevOps 和 SRE 支持,参与包括架构设计、持续交付、上线发布、环境运维、日志监控等工作;
			\item 在 Azure 平台上运行包括生产、灾备环境在内的五套线上环境,使用 Terraform 管理超过一百台虚拟机和其它 Azure 云服务,期间完成在四个月内完成全部环境的 Azure 跨区域迁移和灾备份环境的搭建;
			\item 基于微服务架构,先后使用 Rancher/Kubernetes 作为容器编排平台,期间在三个月的时间里,完成包括生产环境在内的四套环境,从 Rancher 到 Kubernetes 的迁移;
			\item 使用 GoCD,通过 \textit{Pipeline as Code} 方式和 Kubernetes \textit{Elastic Agent} 管理超过 100 条持续交付流水线;
			\item 先后使用 Ansible, Saltstack 管理线上环境配置,保证五套线上环境的一致性和快速变更;
			\item 使用 Prometheus 对线上环境进行全面监控,使用 Elasticsearch 处理和存储约 30GB/天的应用服务日志。
		\end{itemize}
	\end{itemize}
\end{body}

\begin{body}
	\company{乐视致新}{北京}
	{2016年7月 --- 2017年7月}
	{DevOps 运维开发工程师}
	\begin{itemize}[noitemsep,topsep=0pt]
		\item 管理分布式 Gerrit 集群,托管近 1TB 代码库,为全球约 1500 名研发人员提供代码托管服务;
		\item 搭建包括ELK、Superset 在内的实时数据分析平台,收集Gerrit 集群、研发虚拟机集群、代码编译等数据,用于提供数据分析和决策支持;
		\item 构建基于Jenkinsfile的PHP项目持续交付Pipeline;
		\item 编写 Gerrit、ELK 集群的 Ansible Playbook,实现自动化部署与配置更新。
	\end{itemize}
\end{body}

\begin{body}
	\company{BitSight}{剑桥,美国}
	{2015年1月 --- 8月}
	{全职实习运维工程师}
	\begin{itemize}[noitemsep,topsep=0pt]
		\item 为 AWS 上数百台 Ubuntu 集群设计并构建了一套安全补丁管理系统:使用 Sensu 报告可安装补丁,利用 Jenkins 一键安装补丁;
		\item 改进公司持续集成框架:从Jenkins获取测试结果,将测试结果生成报告并发送到BitBucket Pull Request页面;
		\item 使用 Jenkins 和 Fabric 调用 Django API,创建了一个自动化工作流备份和还原 MySQL 测试数据库;
		\item 改进了一些开源项目(包括 automated-ebs-snapshots,BitBucket-api,sensu-community-plugins 等);
		\item 部署并对比几款日志管理系统(ELK, Logentries, Sumologic)使用 Chef 部署,创建控制面板和常用搜索。
	\end{itemize}
\end{body}

\begin{body}
	\company{Locately}{波士顿,美国}
	{2014年7月 --- 8月}
	{全职实习后端软件工程师}
	\begin{itemize}[noitemsep,topsep=0pt]
		\item 实现 AWS 的 Auto Scaling 功能,根据 CPU 使用率动态增减服务器,帮助公司每年节省运营成本 \$3000;
		\item 开发 Apache access log 监视器,将 EC2 实例上收集的数据发送到 AWS CloudWatch,用来监控服务器性能;
		\item 重写 Django App 测试代码,使得 API 测试可以并行运行,从而使测试速度提高 17\%。
	\end{itemize}
\end{body}

%\begin{body}
%	\company{Aston University}{伯明翰,英国}
%	{2012年3月 --- 4月}
%	{实习软件开发}
%	\begin{itemize}[noitemsep,topsep=0pt]
%		\item 开发了一款名为 \textit{Scenario Capture} 的 Eclipse 插件,可以在图形界面下操作,并自动生成 JUnit 测试代码;
%		\item 使用 JUnit 测试代码。
%	\end{itemize}
%\end{body}
\smallskip

%%%%%%%%%%%%%%%%%%%%%%%%%%%%%%%%%%%%%%%%%%%%%%%%%%%%%%%%%%%%%%%%%%%%%%%%%%%%%%%%
% Education
\header{教育经历}

\begin{body}
	\school{\textbf{东北大学},波士顿,美国}{2013年9月 --- 2016年5月}
	{计算机科学硕士学位}{GPA:3.88/4.00}
	% \textbf{部分课程项目}
	% \begin{itemize}
	% \item \project{软件工程项目}{2015年9月 --- 12月}
	% \begin{itemize}
	% 	\item 使用 Jersey 和 Tomcat 构建 JAX-RS 网站:WHAM (What Happened Around Me);
	% 	\item 使用 Jenkins 搭建持续交付工作流;
	% 	\item 根据 Scrum 方法指导项目开发进程,使用 JIRA 进行项目管理和缺陷追踪,Confluence 管理文档。
	% \end{itemize}

	% \item \project{MapReduce 项目}{2015年10月 --- 12月}
	% \begin{itemize}
	% 	\item 统计、分析旧金山市犯罪记录(超过 180 万条记录),并使用朴素贝叶斯算法预测一个案件的犯罪类型;
	% 	\item 编写 Hadoop 和 Spark 程序,在 AWS EMR 上运行。
	% \end{itemize}

	% \item \project{计算机网络项目}{2014年1月 --- 4月}
	% \begin{itemize}
	% 	\item \textit{Raw Sockets} --- 使用 Python 的 Row Socket 构建 TCP/IP 栈;
	% 	\item \textit{Roll Your Own CDN} --- 基于 AWS EC2 搭建一个简易的「内容分发网络(CDN)」。
	% \end{itemize}

	% \item \project{数据库项目}{2013年9月 --- 11月}
	% \begin{itemize}
	% 	\item 使用 JPA 为 MySQL 数据库搭建 OData 服务(一款用来访问数据库的 RESTful API)。
	% \end{itemize}

	% \end{itemize}

	\smallskip
	\school{\textbf{暨南大学},广州}{2009年9月 --- 2013年7月}
	{软件工程工学学士学位}{平均分:84.6/100}
	% \textbf{毕业论文}
	% \begin{itemize}
	% \item 《可视化的Java多线程程序错误定位工具》
	% \begin{itemize}
	% 	\item 实现一个动态错误定位工具,用来定位 Java 多线程程序当中的并发错误;
	% 	\item 使用 \textit{Soot} 分析 Java 字节码,进行插装;SWT 制作图形用户界面。
	% \end{itemize}
	% \end{itemize}
\end{body}

\smallskip

%%%%%%%%%%%%%%%%%%%%%%%%%%%%%%%%%%%%%%%%%%%%%%%%%%%%%%%%%%%%%%%%%%%%%%%%%%%%%%%%
% Skills
\header{专业技能}

\begin{body}
\begin{description}[style=nextline,leftmargin=8em,topsep=1pt]
	\item[语言] Python, Shell, Java, Ruby
	\item[工具] Ansible, Saltstack, Chef, Docker, Terraform, Git, Gerrit, Jenkins, GoCD, MySQL
	\item[云服务] Azure, AWS
	\item[监控] Prometheus, Elastic Stack, Nagios
\end{description}
\end{body}

%\smallskip

% \newpage{} % uncomment this line if you want to force a new page


%%%%%%%%%%%%%%%%%%%%%%%%%%%%%%%%%%%%%%%%%%%%%%%%%%%%%%%%%%%%%%%%%%%%%%%%%%%%%%%%
% Publications
%\header{Selected Publications}

%\begin{body}
%	\vspace{14pt}
%	Seth Holloway. ``\textbf{Writing Resumes in LaTeX},'' in \emph{Proceedings of the Mind of Seth Holloway}. 2011.\\
%	\smallskip
%	Seth Holloway, Andrea Holloway. ``\textbf{Accent Modification Rules},'' in \emph{Proceedings of Fun}. 2008.\\
%\end{body}
%
%\smallskip

%%%%%%%%%%%%%%%%%%%%%%%%%%%%%%%%%%%%%%%%%%%%%%%%%%%%%%%%%%%%%%%%%%%%%%%%%%%%%%%%
% Awards and Honors
%\header{AWARDS AND HONORS}

%\begin{body}
%	\textbf{Third Class University Scholarship}, Jinan University \hfill{} 2011\\
%	\smallskip
%	\textbf{Third Class Graduating Student Scholarship},  Jinan University \hfill{} 2013
%\end{body}

\header{专业认证}

\begin{body}
	\begin{itemize}[noitemsep,topsep=0pt]
		\item AWS Certified Solutions Architect(Professional)
		\item Certified Kubernetes Administrator
	\end{itemize}
\end{body}

%\smallskip

%%%%%%%%%%%%%%%%%%%%%%%%%%%%%%%%%%%%%%%%%%%%%%%%%%%%%%%%%%%%%%%%%%%%%%%%%%%%%%%%
% Interests
\header{兴趣爱好}
\begin{body}
	摄影,旅行,足球
\end{body}

\end{document}
